% Copyright (C) Data Structures and Algorithms Team.
\chapter*{Preface}
Every book has a story as to how it came about and this one is no different, however we would be lying if we said its development was not somewhat impromptu. Put simply this book is a result of a series of emails sent between the two authors during the development of a library for the .NET framework of the same name (with the omission of the subtitle). The conversation started off something like ``Why don't we create a more aesthetically pleasing way to present the pseudo code that we have?'' After a few weeks this new aesthetically pleasing way to show our pseudo code had in fact grown into pseudo code listings with chunks of text describing how it works and various other things and so we thought ``what the heck, lets make this thing into a book!'' And so, in the summer of 2008 we began work on this book side by side with the actual implementation.

When we started writing this book the only things that we were sure about with respect to how the book should be structured were:
\begin{inparaenum}
\item always make explanations as simple as possible while maintaining a moderatley atomic degree of precision to keep the more eager minded happy; and
\item inject diagrams to demystify problems that are even moderatly challenging to visualise (...and so we could remember how our own algorithms worked when looking back at them!); and finally
\item present concise and self explanatory pseudo code listings that can be ported easily to most mainstream imperative programming languages like C++, C\#, and Java.
\end{inparaenum}

A key factor of this book and its associated implementation are that all algorithms (unless otherwise stated) were designed by us (using the theory of the algorithm in question as a guideline, for which we are eternally grateful to their original creators) they may turn out to be worse than the ``normal'' implementations, they may not. We are two fellows of the opinion that choice is a great thing. Read our book, read several others on the same subject and use what you see fit from each (if anything) when implementing your own version of the algorithm(s) in question.

We very much hope that you enjoy our book on what is a challenging, but very rewarding subject and hope that it helps you (at least somewhat) in the near future.

% insert acknowledgements here later

\begin{center}
Granville Barnett \\
Luca Del Tongo
\end{center}
