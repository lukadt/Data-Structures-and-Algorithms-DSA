% Copyright (C) Data Structures and Algorithms Team.
\chapter*{Preface}
Throughout the history of programming efficiency has always been key, with respect to both time (run time - for which we use big Oh notation) and space (memory) this makes choosing the correct data structures and algorithms in a project vital to it's success, the wrong selection can result in adverse affects - worse it may lose you your customer base!

The premise of this book is to fill the void, to provide information that allows you to select the right data structures and algorithms within your given context. While it is attributed that we cannot give you one-to-one counselling we can, however, provide you with the required knowledge in order for you to make educated decisions.

Throughout this book you will find annotations littered with advice that we have found useful with respect to that algorithm or data structure - this advice is based on our own experiences and at times no doubt it is very subjective, but it should (hopefully!) still provoke thought from the reader.

Most of the work in this book is based on the work we carried out on the Data Structures and Algorithms (DSA)\footnote{\url{http://codeplex.com/dsa}} library. DSA was built using C\# and the .NET platform and serves as a project for all the things that we felt .NET was missing with respect to core data structures and algorithms - choice is imperative in all but simple scenarios, this itself was a driving factor, we wanted to provide the modern programmer with more choice.

\section*{Intended Use}
This book serves purely as a reference for those who are wanting to get a quick overview of the data structures and algorithms listed, as well as having a tried and tested implementation in the form of pseudo code to port to the readers imperative\footnote{You can also use this book for functional languages, however you will need to inject idioms that are core to that paradigm to create well structured functional alternatives of the work listed in this book.} language of choice.

The implementations in this book have straight forward translations to the following languages:

\begin{enumerate}
\item C++
\item C\#
\item Java
\end{enumerate}

Most of the algorithms in this book do not rely on algorithms or data structures provided by mainstream libraries (e.g. Base Class Library (BCL) in .NET, or Standard Template Library (STL) in C++) unless we are sure that such properties exist across the intended language libraries.
