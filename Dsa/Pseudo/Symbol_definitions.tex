% Copyright (C) Data Structures and Algorithms Team.
\chapter{Symbol Definitions}
Throughout the pseudocode listings you will find several symbols used,  describes the meaning of each of those symbols.

\begin{table}[h]
\caption{Pseudo symbol definitions}
\begin{tabular}[t]{|c|l|}
\hline
\textbf{Symbol} & \textbf{Description} \\
\hline
$\leftarrow$ & Assignment. In most imperative languages this equates to $=$ \\
\hline
$=$ & Equality. Most imperative languages use $==$ to denote a check for equality. \\
\hline
$\leq$ & Less than or equal to. Commonly this is denoted as $<=$ in imperative languages. \\
\hline
$<$ & Less than.* \\
\hline
$\geq$ & Greater than or equal to. Commonly this is denoted as $>=$ in imperative languages. \\
\hline
$>$ & Greater than.* \\
\hline
$!=$ & Inequality. * \\
\hline
$\emptyset$ & Null. \\
\hline
\textbf{and} & Logical and. Commonly \&\& in imperative languages. \\
\hline
\textbf{or} & Logical or. Commonly || in imperative languages. \\ 
\hline
whitespace & Correct token in your language that represents whitespace. \\
\hline
\textbf{yield} & Like \textbf{return} but builds a sequence. \\
\hline
\end{tabular}
\end{table}

* This symbol has a direct translation with the vast majority of imperative counterparts.
