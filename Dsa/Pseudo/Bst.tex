

% Copyright (C) Data Structures and Algorithms Team.
\chapter{Binary Search Tree}
Binary search tree's (BSTs) are very simple to understand, consider the following where by we have a root node $n$, the left sub tree of $n$ contains values $< n$, the right sub tree however contains nodes whose values are $\geq n$.

BSTs are of interest because they have operations which are favourably fast, insertion, look up, and deletion can all be done in $O(log~n)$. One of the things that I would like to point out and address early is that $O(log~n)$ times for the aforementioned operations can only be attained if the BST is relatively balanced (for a tree data structure with self balancing properties see \S\ref{AVL}). 

\section{Insertion}
As mentioned previously insertion is an $O(log~n)$ operation provided that the tree is moderately balanced.

\begin{tabbing}
1)  \textbf{alg}\= \textbf{orithm} Insert($value$) \\
2)  \> \textbf{Pre:}~~$value$ has passed custom type checks for type $T$ \\
3)  \> \textbf{Post:}~$value$ has been placed in the correct location in the tree \\
4)  \> \textbf{if}~\= $root~= \emptyset$ \\
5)  \> \> $root \leftarrow$ node($value$) \\
6)  \> \textbf{else} \\
7)  \> \> InsertNode($root$, $value$) \\
8)  \> \textbf{end if} \\
9)  \textbf{end} Insert \\
\end{tabbing}

\begin{tabbing}
1)  \textbf{alg}\= \textbf{orithm} InsertNode($root$, $value$) \\
2)  \> \textbf{Pre:}~~$root$ is the node to start from \\
3)  \> \textbf{Post:}~$value$ has been placed in the correct location in the tree \\
4)  \> \textbf{if}~\= $value < root$.Value \\
5)  \> \> \textbf{if}~\= $root$.Left $= \emptyset$ \\
6)  \> \> \> $root$.Left $\leftarrow$ node($value$) \\
7)  \> \> \textbf{else} \\
8)  \> \> \> InsertNode($root$.Left, $value$) \\
9)  \> \> \textbf{end if} \\
10) \> \textbf{else} \\
11) \> \> \textbf{if} $root$.Right $= \emptyset$ \\
12) \> \> \> $root$.Right $\leftarrow$ node($value$) \\
13) \> \> \textbf{else} \\
14) \> \> \> InsertNode($root$.Right, $value$) \\
15) \> \> \textbf{end if} \\
16) \> \textbf{end if} \\
17) \textbf{end} InsertNode \\ 
\end{tabbing}

The insertion algorithm is split for a good reason, the first algorithm (non-recursive) checks a very core base case - whether or not the tree is empty, if the tree is empty then we simply create our root node and we have no need to invoke the recursive $InsertNode$ algorithm. When the core base case is not met we must invoke the recursive $InsertNode$ algorithm which simply guides us to the first appropriate place in the tree to put $value$.

\section{Searching}
Searching a BST is really quite simple, the pseudo code is self explanatory but I will explain briefly the premise of the algorithm nonetheless.

We have talked previously about insertion, we go either left or right with the right sub tree containing values that are $\geq n$ where $n$ is the value of the node we are inserting, when searching the rules are made a little more atomic - given $n$ we inspect the value of the current node $c$ if $n < c$ then we go left, if $n >c$ we go right, if it is neither ($n = c$) we have found the value. The base case to this algorithm (which is recursive) is that the $root$ node is $\emptyset$, in such an event we can report back to the caller that $n$ is not in the BST as we have ran out of nodes in the tree to search.

\begin{tabbing}
1)  \textbf{alg}\= \textbf{orithm} Contains($root$, $value$) \\
2)  \> \textbf{Pre:}~~$root$ is the root node of the tree, $value$ is what we would like to locate \\
3)  \> \textbf{Post:}~$value$ is either located or not \\
4)  \> \textbf{if}~\= $root = \emptyset$ \\
5)  \> \> \textbf{return false} \\
6)  \> \textbf{end if} \\
7)  \> \textbf{if} $root$.Value $= value$ \\
8)  \> \> \textbf{return true} \\
9)  \> \textbf{else if} $value < root$.Value \\
10) \> \> \textbf{return} Contains($root$.Left, $value$) \\
11) \> \textbf{else} \\
12) \> \> \textbf{return} Contains($root$.Right, $value$) \\
13) \> \textbf{end if} \\
14) \textbf{end} Contains \\
\end{tabbing}

\section{Deletion}
Removing a node from a BST is simple, there are $4$ cases that we must consider though: 
\begin{inparaenum}
\item the value to remove is a leaf node; or
\item the value to remove has a right sub tree, but no left sub tree; or
\item the value to remove has a left sub tree, but no right sub tree; or
\item the value to remove has both a left and right sub tree in which case we promote the largest value in the left sub tree.
\end{inparaenum}
The $Remove$ algorithm described later relies on two further helper algorithms named $FindParent$, and $FindNode$ which are described in \S\ref{finding_parent_node} and \S\ref{find_node_reference}.

\begin{tabbing}
1)  \textbf{alg}\= \textbf{orithm} Remove($value$) \\
2)  \> \textbf{Pre:}~~$value$ is the value of the node to remove, $root$ is the root node of the BST \\
3)  \> \textbf{Post:}~node with $value$ is removed if found in which case yields true, otherwise false \\
4)  \> $nodeToRemove \leftarrow$ FindNode($value$) \\
5)  \> \textbf{if}~\= $nodeToRemove = \emptyset$ \\
6)  \> \> \textbf{return false} // value not in BST \\
7)  \> \textbf{end if} \\
8)  \> $parent \leftarrow$ FindParent($value$) \\
9)  \> \textbf{if} $count = 1$ // $count$ keeps track of the \# of nodes in the BST \\
10) \> \> $root \leftarrow \emptyset$ // we are removing the only node in the BST \\
11) \> \textbf{else if} $nodeToRemove$.Left $= \emptyset$ \textbf{and} $nodeToRemove$.Right $= null$ \\
12) \> \> // case \#1 \\
13) \> \> \textbf{if}~\= $nodeToRemove$.Value $< parent$.Value \\
14) \> \> \> $parent$.Left $\leftarrow \emptyset$ \\
15) \> \> \textbf{else} \\
16) \> \> \> $parent$.Right $\leftarrow \emptyset$ \\
17) \> \> \textbf{end if} \\
18) \> \textbf{else if} $nodeToRemove$.Left $= \emptyset$ \textbf{and} $nodeToRemove$.Right $!= \emptyset$ \\
19) \> \> // case \# 2 \\
20) \> \> \textbf{if} $nodeToRemove$.Value $< parent$.Value \\
21) \> \> \> $parent$.Left $\leftarrow nodeToRemove$.Right \\
22) \> \> \textbf{else} \\
23) \> \> \> $parent$.Right $\leftarrow nodeToRemove$.Right \\
24) \> \> \textbf{end if} \\
25) \> \textbf{else if} $nodeToRemove$.Left $!= \emptyset$ \textbf{and} $nodeToRemove$.Right $= \emptyset$ \\
26) \> \> // case \#3 \\
27) \> \> \textbf{if} $nodeToRemove$.Value $< parent$.Value \\
28) \> \> \> $parent$.Left $\leftarrow nodeToRemove$.Left \\
29) \> \> \textbf{else} \\
30) \> \> \> $parent$.Right $\leftarrow nodeToRemove$.Left \\
31) \> \> \textbf{end if} \\
32) \> \textbf{else} \\
33) \> \> // case \#4 \\
34) \> \> $largestValue \leftarrow nodeToRemove$.Left \\
35) \> \> \textbf{while} $largestValue$.Right $!= \emptyset$ \\
36) \> \> \> // find the largest value in the left sub tree of $nodeToRemove$ \\
37) \> \> \> $largestValue \leftarrow largestValue$.Right \\
38) \> \> \textbf{end while} \\
39) \> \> // set the parents' Right pointer of $largestValue$ to $\emptyset$ \\
40) \> \> FindParent($largestValue$.Value).Right $\leftarrow \emptyset$ \\
41) \> \> $nodeToRemove$.Value $\leftarrow largestValue$.Value \\
42) \> \textbf{end if} \\
43) \> $count \leftarrow count - 1$ \\
44) \> \textbf{return true} \\
45) \textbf{end} Remove \\
\end{tabbing}

\section{Finding the parent of a given node} \label{finding_parent_node}
The purpose of this algorithm is simple - to return a reference (or a pointer) to the parent node of the node with the given value. We have found that such an algorithm is very useful, especially when performing extensive tree transformations.

\begin{tabbing}
1)  \textbf{alg}\= \textbf{orithm} FindParent($value$, $root$) \\
2)  \> \textbf{Pre:}~~$value$ is the value of the node we want to find the parent of \\
3)  \> ~~~~~~~~$root$ is the root node of the BST and is $!= \emptyset$ \\
4)  \> \textbf{Post:}~a reference to the parent node of $value$ if found; otherwise $\emptyset$ \\
5)  \> \textbf{if}~\= $value = root$.Value \\
6)  \> \> \textbf{return} $\emptyset$ \\
7)  \> \textbf{end if} \\
8)  \> \textbf{if} $value < root$.Value \\
9)  \> \> \textbf{if}~\= $root$.Left $= \emptyset$ \\
10) \> \> \> \textbf{return} $\emptyset$ \\
11) \> \> \textbf{else if} $root$.Left.Value $= value$ \\
12) \> \> \> \textbf{return} $root$ \\
13) \> \> \textbf{else} \\
14) \> \> \> \textbf{return} FindParent($value$, $root$.Left) \\
15) \> \> \textbf{end if} \\
16) \> \textbf{else} \\
17) \> \> \textbf{if} $root$.Right $= \emptyset$ \\
18) \> \> \> \textbf{return} $\emptyset$ \\
19) \> \> \textbf{else if} $root$.Right.Value $= value$ \\
20) \> \> \> \textbf{return} $root$ \\
21) \> \> \textbf{else} \\
22) \> \> \> \textbf{return} FindParent($value$, $root$.Right) \\
23) \> \> \textbf{end if} \\
24) \> \textbf{end if} \\
25) \textbf{end} FindParent \\
\end{tabbing}

\section{Attaining a reference to a node} \label{find_node_reference}
Just like the algorithm explained in \S\ref{finding_parent_node} this algorithm we have found to be very useful, it simply finds the node with the specified value and returns a reference to that node.

\begin{tabbing}
1)  \textbf{alg}\= \textbf{orithm} FindNode($value$, $root$) \\
2)  \> \textbf{Pre:}~~$value$ is the value of the node we want to find the parent of \\
3)  \> ~~~~~~~~$root$ is the root node of the BST \\
4)  \> \textbf{Post:}~a reference to the node of $value$ if found; otherwise $\emptyset$ \\
5)  \> \textbf{if}~\= $root = \emptyset$ \\
6)  \> \> \textbf{return} $\emptyset$ \\
7)  \> \textbf{end if} \\
8)  \> \textbf{if} $value < root$.Value \\
9)  \> \> \textbf{return} FindNode($value$, $root$.Left) \\
10) \> \textbf{else if} $value > root$.Value \\
11) \> \> \textbf{return} FindNode($value$, $root$.Right) \\
12) \> \textbf{else} \\
13) \> \> \textbf{return} $root$ \\
14) \> \textbf{end if} \\
15) \textbf{end} FindNode \\
\end{tabbing}
