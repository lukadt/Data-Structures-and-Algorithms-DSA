% Copyright (C) Data Structures and Algorithms Team.
\chapter{Binary Search Tree}
Binary search tree's (BSTs) are very simple to understand, consider the following where by we have a root node $n$, the left sub tree of $n$ contains values $< n$, the right sub tree however contains nodes whose values are $\geq n$.

BSTs are of interest because they have operations which are favourably fast, insertion, look up, and deletion can all be done in $O(log~n)$. One of the things that I would like to point out and address early is that $O(log~n)$ times for the aforementioned operations can only be attained if the BST is relatively balanced (for a tree data structure with self balancing properties see \S\ref{AVL}). 

\section{Insertion}
As mentioned previously insertion is an $O(log~n)$ operation provided that the tree is moderately balanced.

\begin{tabbing}
1)  \textbf{alg}\= \textbf{orithm} Insert($value$) \\
2)  \> \textbf{Pre:}~~$value$ has passed custom type checks for type $T$ \\
3)  \> \textbf{Post:}~$value$ has been placed in the correct location in the tree \\
4)  \> \textbf{if}~\= $root~= \emptyset$ \\
5)  \> \> $root \leftarrow$ node($value$) \\
6)  \> \textbf{else} \\
7)  \> \> InsertNode($root$, $value$) \\
8)  \> \textbf{end if} \\
9)  \textbf{end} Insert \\
\end{tabbing}

\begin{tabbing}
1)  \textbf{alg}\= \textbf{orithm} InsertNode($root$, $value$) \\
2)  \> \textbf{Pre:}~~$root$ is the node to start from \\
3)  \> \textbf{Post:}~$value$ has been placed in the correct location in the tree \\
4)  \> \textbf{if}~\= $value < root$.Value \\
5)  \> \> \textbf{if}~\= $root$.Left $= \emptyset$ \\
6)  \> \> \> $root$.Left $\leftarrow$ node($value$) \\
7)  \> \> \textbf{else} \\
8)  \> \> \> InsertNode($root$.Left, $value$) \\
9)  \> \> \textbf{end if} \\
10) \> \textbf{else} \\
11) \> \> \textbf{if} $root$.Right $= \emptyset$ \\
12) \> \> \> $root$.Right $\leftarrow$ node($value$) \\
13) \> \> \textbf{else} \\
14) \> \> \> InsertNode($root$.Right, $value$) \\
15) \> \> \textbf{end if} \\
16) \> \textbf{end if} \\
17) \textbf{end} InsertNode \\ 
\end{tabbing}

The insertion algorithm is split for a good reason, the first algorithm (non-recursive) checks a very core base case - whether or not the tree is empty, if the tree is empty then we simply create our root node and we have no need to invoke the recursive $InsertNode$ algorithm. When the core base case is not met we must invoke the recursive $InsertNode$ algorithm which simply guides us to the first appropriate place in the tree to put $value$.
