% Copyright (C) Data Structures and Algorithms Team.
\chapter{Numeric}
Unless stated otherwise the alias $n$ denotes a standard $32$ bit integer.

\section{Primality Test} \label{cha:Primality}
A simple algorithm that determines whether or not a given integer is a prime number, e.g. $2$, $5$, $7$, and $13$ are \textbf{all} prime numbers, however $6$ is not as it can be the result of the product of two numbers that are $< 6$.

In an attempt to slow down the inner loop the $\sqrt{n}$ is used as the upper bound.

\begin{tabbing}
1) \textbf{alg}\= \textbf{orithm} IsPrime($n$)\\
2) \> \textbf{Post:} $n$ is determined to be a prime or not \\
3) \> \textbf{for} \= $i \leftarrow 2$ \textbf{to} $n$ \textbf{do}\\
4) \> \> \textbf{for} \= $j \leftarrow 1$ \textbf{to} $sqrt(n)$ \textbf{do}\\
5) \> \> \> \textbf{if}~\= $i * j = n$\\
6) \> \> \> \> \textbf{return} false\\
7) \> \> \> \textbf{end if}\\
8) \> \> \textbf{end for}\\	
9) \> \textbf{end for}\\
10) \textbf{end} IsPrime
\end{tabbing}

\section{Base conversions}
DSA contains a number of algorithms that convert a base $10$ number to its equivalent binary, octal or hexadecimal form. For example $78_{10}$ has a binary representation of $1001110_{2}$.

Table \ref{tab:tobinary_trace} shows the algorithm trace when the number to convert to binary is $742_{10}$.

\newpage
\begin{tabbing}
1) \textbf{alg}\= \textbf{orithm} ToBinary($n$)\\
2) \> \textbf{Pre:}~~$n \geq 0$ \\
3) \> \textbf{Post:}~$n$ has been converted into its base $2$ representation \\
4) \> \textbf{whi}\= \textbf{le} $n > 0$\\
5) \> \> $list.Add(n~\%~2)$\\
6) \> \> $n \leftarrow n / 2$\\
7) \> \textbf{end while}\\
8) \> \textbf{return} Reverse($list$) \\
9) \textbf{end} ToBinary\\
\end{tabbing}

\begin{table}[h]
\begin{center}
\begin{tabular}{|l|l|}
\hline
$n$ & $list$ \\
\hline
$742$ & \{ $0$ \} \\
\hline
$371$ & \{ $0, 1$ \} \\
\hline
$185$ & \{ $0, 1, 1$ \} \\
\hline
$92$ & \{ $0, 1, 1, 0$ \} \\
\hline
$46$ & \{ $0, 1, 1, 0, 1$ \} \\
\hline
$23$ & \{ $0, 1, 1, 0, 1, 1$ \} \\
\hline
$11$ & \{ $0, 1, 1, 0, 1, 1, 1$ \} \\
\hline
$5$ & \{ $0, 1, 1, 0, 1, 1, 1, 1$ \} \\
\hline
$2$ & \{ $0, 1, 1, 0, 1, 1, 1, 1, 0$ \} \\
\hline
$1$ & \{ $0, 1, 1, 0, 1, 1, 1, 1, 0, 1$ \} \\
\hline
\end{tabular}
\end{center}
\caption{Algorithm trace of ToBinary} \label{tab:tobinary_trace}
\end{table}

\section{Attaining the greatest common denominator of two numbers}
A fairly routine problem in mathematics is that of finding the greatest common denominator of two integers, what we are essentially after is the greatest number which is a multiple of both, e.g. the greatest common denominator of $9$, and $15$ is $3$. One of the most elegant solutions to this problem is based on Euclid's algorithm that has a run time complexity of $O(n^{2})$.

\begin{tabbing}
1) \textbf{alg}\= \textbf{orithm} GreatestCommonDenominator($m$, $n$)\\
2) \> \textbf{Pre:}~~$m$ and $n$ are integers \\
3) \> \textbf{Post:}~the greatest common denominator of the two integers is calculated \\
4) \> \textbf{if}~\= $n = 0$ \\
5) \> \> \textbf{return} $m$ \\
6) \> \textbf{end if}\\
7) \> \textbf{return} GreatestCommonDenominator($n$, $m~\%~n$) \\
8) \textbf{end} GreatestCommonDenominator\\
\end{tabbing}

