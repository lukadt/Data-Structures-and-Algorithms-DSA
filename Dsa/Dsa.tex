\documentclass[10pt,oneside,a4paper]{report}
\usepackage{url}
\begin{document}

\title{Data Structures and Algorithms\\Pseudocode}
\author{Granville Barnett\\Luca Del Tongo}
\maketitle

\newpage
\tableofcontents
\newpage

\section*{Preface}
This book consists of annotated pseudo code that was used when designing the algorithms which are contained in DSA\footnote{\url{http://codeplex.com/dsa}}. In DSA all implementations are in C\# \footnote{\url{http://msdn.microsoft.com/en-us/vcsharp/default.aspx}}.

The designs provided here serve as a reference for those who want to port well designed and tested implementations of common (and uncommon) data structures and algorithms to their imperative language of choice.

\pagestyle{headings}

\part{Data Structures}

\part{Algorithms}

\chapter{Sorting}
Test.

\chapter{Numeric}
\section{IsPrime}
A simple algorithm that determines whether or not a given integer is a prime number, e.g. $2$, $5$, $7$, and $13$ are \textbf{all} prime numbers, however $6$ is not as it can be the result of the product of two numbers that are $< 6$.

In an attempt to slow down the inner loop the $\sqrt{n}$ is used as the upper bound.
\begin{tabbing}
1) \textbf{alg}\= \textbf{orithm} IsPrime($n$)\\
2) \> \textbf{for} \= $i \leftarrow 2$ \textbf{to} $n$ \textbf{do}\\
3) \> \> \textbf{for} \= $j \leftarrow 1$ \textbf{to} $sqrt(n)$ \textbf{do}\\
4) \> \> \> \textbf{if} ~\= $i * j = n$\\
5) \> \> \> \> \textbf{return} false\\
6) \> \> \> \textbf{end if}\\
7) \> \> \textbf{end for}\\	
8) \> \textbf{end for}\\
9) \textbf{end} IsPrime
\end{tabbing}

\section{ToBaseN}
DSA contains a number of algorithms that convert a base $10$ number to its equivalent binary, octal or hexadecimal form.

\begin{tabbing}
1) \textbf{alg}\= \textbf{orithm} ToBinary($n$)\\
2) \> \textbf{whi}\= \textbf{le} $n > 0$\\
3) \> \> $list.Add(n~\%~2)$\\
4) \> \> $n \leftarrow n / 2$\\
5) \> \textbf{end while}\\
6) \textbf{end} ToBinary\\
\end{tabbing}

\end{document}
